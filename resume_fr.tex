\documentclass[10pt,a4paper]{article}
\usepackage[margin=3em]{geometry}
\usepackage{mdwlist}
\usepackage{color}
\usepackage{hyperref}
\usepackage{graphicx}
\usepackage{xspace}
\usepackage{datetime}

\newdateformat{frenchDate}{\THEDAY~\monthname,~\THEYEAR}

\newcommand{\Csharp}{%
  {\settoheight{\dimen0}{C}C\kern-.05em \resizebox{!}{\dimen0}{\raisebox{\depth}{\#}}}}

\newcommand{\latex}{\LaTeX\xspace}

% Configuration.
\pagestyle{empty}
\setlength{\tabcolsep}{0em}

% indentsection style, used for sections that aren't already in lists
% that need indentation to the level of all text in the document
\newenvironment{indentsection}[1]
{\begin{list}{}
  {\setlength{\leftmargin}{#1}} \item[]
}
{\end{list}}

% opposite of above; bump a section back toward the left margin
\newenvironment{unindentsection}[1]
{\begin{list}{}
  {\setlength{\leftmargin}{-0.5#1}} \item[]
}
{\end{list}}
% format two pieces of text, one left aligned and one right aligned
\newcommand{\headerrow}[2]
{\begin{tabular*}{\linewidth}{l@{\extracolsep{\fill}}r}
  #1 &
  #2 \\
\end{tabular*}}


\begin{document}


% Resume header.
\begin{center}
  \Large\textbf{Victor Hachard} \normalsize\emph{(né en 1999)} \\
  \large\emph{première année de master (passerelle)} \\
  \normalsize\href{mailto:victor.hachard@hotmail.fr}{victor.hachard@hotmail.fr}
  $\circ$
  0472 43 39 08 \\
  Rue Childéric 18, 7500 Tournai \\
  \href[pdfnewwindow=true]{http://www.github.com/VictorHachard}{github.com/VictorHachard}
  $\circ$
  \href[pdfnewwindow=true]{http://www.victorhachard.fr}{victorhachard.fr}
  \vspace{-0.2em}
\end{center}


% Education section.
\vspace{-0.4em}
\subsection*{Formations}
\begin{itemize}
  \parskip=0.1em

  \item
  \headerrow
    {\textbf{Bachelier -- Informatique et Systèmes} orientation Réseaux et Télécommunications}
    {\emph{2018 -- 2021}}
  \headerrow
    {{Haute École en Hainaut, Mons}}
    {\emph{}}
  \begin{itemize*}
    \setlength\itemsep{0.5em}
    \item[]
    Cursus : télécommunications et réseaux, développement web, conception
    orientée objet, algorithmique, bases de données relationnelles, conception d'applications.
    \item[]
    Stage de 13 semaines dans l'entreprise Technord, j'ai réalisé un installateur avec NSIS et une application web avec Angular...
  \end{itemize*}
  \item
  \headerrow
    {\textbf{Bachelier -- Sciences informatiques}}
    {\emph{2017 -- 2018}}
  \headerrow
    {{Umons, Mons}}
    {\emph{}}
  \begin{itemize*}
    \item[]
    Cursus : fonctionnement des ordinateurs, programmation et algorithmique, physique,
    mathématiques.
   \item[] Durant ce cursus j'ai appris énormément en programmation, ce qui a confirmé ma
   passion donc mes études.
  \end{itemize*}
  \item
  \headerrow
    {\textbf{Technique de qualification -- Arts visuels et Photographie}}
    {\emph{2015 -- 2017}}
  \headerrow
    {{Institut Saint-Luc, Tournai}}
    {\emph{}}
  \begin{itemize*}
    \item[]
    CESS, Certificat de technicien en photographie, Certificat de gestion de
    base.
    \item[]
    Durant ce cursus je suis devenu plus minutieux et plus rigoureux, c'est l'attention au détail qui fait la photo magnifique.
    \item[]
    \rule{0pt}{2ex}\footnotesize Mes photos @ \href[pdfnewwindow=true]{http://www.victorhachard.fr}
    {victorhachard.fr}
  \end{itemize*}
\end{itemize}


% Projects section.
\vspace{-0.4em}
\subsection*{Travaux réalisés/en cours}
\begin{itemize}
  \parskip=0.1em

  \item
    \headerrow
        {\textbf{\href[pdfnewwindow=true]{http://www.github.com/VictorHachard/YPC\textunderscore2.0}
      {YouTube Playlist Checker}}}
        {\emph{2018 -- 2020}}
    \headerrow
      {\emph{Travail personnel en cours}}
      {\emph{}}
    \begin{itemize*}
      \item[]
      Création d'une application web : \href{http://www.ypc.yt}{ypc.yt} permet d'archiver
      des vidéos YouTube. Apprentisage : PHP, SQL, structuration d'une base de données,
      création d'une application web/API depuis zéro, utilisation de L'API YouTube.
    \end{itemize*}
    \item
    \headerrow
      {\textbf{\href[pdfnewwindow=true]{http://www.github.com/VictorHachard/GameEngine}
        {Game Engine}}}
      {\emph{2018}}
    \headerrow
      {\emph{Travail personnel en cours Soon\texttrademark}}
      {\emph{}}
      \begin{itemize*}
        \item[]
        Création d'un moteur de jeu simple. Apprentisage : Java/JavaFX avancée,
        organisation/architecture d'un projet de taille moyenne.
      \end{itemize*}
    \item
    \headerrow
      {\textbf{ \href[pdfnewwindow=true]{http://www.github.com/VictorHachard/BobyIsYou}
        {Boby Is You}}}
      {\emph{2017}}
    \headerrow
      {\emph{Travail scolaire réalisé}}
      {\emph{}}
      \begin{itemize*}
        \item[]
        Remake du jeu Baba Is You de Hempuli.
        Apprentisage : Java, JavaFX, gestion de fichiers, Apache Ant.
      \end{itemize*}
  \end{itemize}


% Experience section.
\vspace{-0.4em}
\subsection*{Expériences extra-scolaires}
\begin{itemize}
  \parskip=0.1em

  \item
  \headerrow
    {\textbf{Job étudiant} -- rayon boucherie}
    {\emph{2019 -- 2020}}
  \headerrow
    {{Carrefour EU, Froyennes}}
    {\emph{}}
  \begin{itemize*}
    \item[]
    Durant cette expérience j'ai pu m'améliorer en communication avec les contacts clients ainsi que me familiariser
    avec le monde du travail en grande entreprise.
  \end{itemize*}
  \item
  \headerrow
    {\textbf{Stagiaire éclairagiste}}
    {\emph{2012 -- 2015}}
  \headerrow
    {{Jet Sound, H\&D Technologie, Opéra Bastille, Belgique et France}}
    {\emph{}}
  \begin{itemize*}
    \item[]
    Ces stages étaient superbes : j'étais un enfant réalisant ses rêves. Le meilleur moment pour moi était de
    contrôler l'éclairage en live durant une soirée.
  \end{itemize*}
\end{itemize}


% Skills section.
\vspace{-0.4em}
\subsection*{Compétences}
\begin{indentsection}{\parindent}
\begin{description*}
  \item[Langues :]
  Français \emph{(langue maternelle)}, %anglais informatique \emph{(lu, parlé, écrit)},
  Anglais \emph{(bonnes connaissances)}.
  \item[Langages :] Java, Python, PHP, SQL, C, \Csharp, HTML,
  CSS, JavaScript/jQuery, TypeScript, Shell, Bash.
  \item[Technologies :] Spring Boot, Angular.
  \item[Systèmes :] Windows, Windows Server, Linux.
  \item[Logiciels :] IntelliJ, WebStorm, Eclipse, Visual Studio Code, Git,
  Photoshop.
\end{description*}
\end{indentsection}


% Interests/About section.
\begin{indentsection}{\parindent}
\begin{description*}
\item[Centres d'intérêt :]
  électricité, électronique, photographie, jeux vidéo (factory building) et natation.
\end{description*}
\end{indentsection}


% Metalink.
\begin{center}
\footnotesize \latex source code @
\href{http://www.github.com/VictorHachard/resume}
{github.com/VictorHachard/resume} - fork @
\href{http://www.github.com/nixpulvis}
{github.com/nixpulvis} \\
% dernière mise à jour \frenchDate\today
\end{center}
\end{document}
