\documentclass[10pt,letterpaper]{article}
\usepackage[letterpaper,margin=3em]{geometry}
\usepackage{mdwlist}
\usepackage{color}
\usepackage{hyperref}
\usepackage{graphicx}
\usepackage{xspace}

\newcommand{\Csharp}{%
  {\settoheight{\dimen0}{C}C\kern-.05em \resizebox{!}{\dimen0}{\raisebox{\depth}{\#}}}}

\newcommand{\latex}{\LaTeX\xspace}

% Configuration.
\pagestyle{empty}
\setlength{\tabcolsep}{0em}

% indentsection style, used for sections that aren't already in lists
% that need indentation to the level of all text in the document
\newenvironment{indentsection}[1]
{\begin{list}{}
  {\setlength{\leftmargin}{#1}} \item[]
}
{\end{list}}

% opposite of above; bump a section back toward the left margin
\newenvironment{unindentsection}[1]
{\begin{list}{}
  {\setlength{\leftmargin}{-0.5#1}} \item[]
}
{\end{list}}
% format two pieces of text, one left aligned and one right aligned
\newcommand{\headerrow}[2]
{\begin{tabular*}{\linewidth}{l@{\extracolsep{\fill}}r}
  #1 &
  #2 \\
\end{tabular*}}


\begin{document}

% Resume header.
\begin{center}
  \LARGE\textbf{Victor Hachard} \normalsize\emph{6 Juin, 1999} \\
  \href{mailto:victor.hachard@hotmail.fr}{victor.hachard@hotmail.fr}
  $\circ$
  0472 43 39 08
  $\circ$
  \href{http://www.github.com/VictorHachard}{github.com/VictorHachard}
  $\circ$
  \href[pdfnewwindow=true]{http://www.victorhachard.fr}{victorhachard.fr}
  \\ 7500 Tournai, Belgique
  \vspace{-0.2em}
\end{center}


% Experience section.
\hrule
\vspace{-0.4em}
\subsection*{Expériences Professionnelles}
\begin{itemize}
  \parskip=0.1em

  \item
  \headerrow
    {\textbf{Carrefour EU}}
    {\textbf{Froyennes, Belgique}}
  \headerrow
    {\emph{Job étudiant}}
    {\emph{Juin 2019 -- September 2020}}
  \end{itemize}


% Education section.
\hrule
\vspace{-0.4em}
\subsection*{Formations}
\begin{itemize}
  \parskip=0.1em

  \item
  \headerrow
    {\textbf{Haute École en Hainaut}}
    {\textbf{Mons, Belgique}}
  \headerrow
    {\emph{But : Bachelier en Informatique et Systèmes orientation Réseaux et
    Télécommunications}}
    {\emph{2018 -- 2020}}
    \emph{Actuellement en troisième année.}
  \begin{itemize*}
    \item
    Ressources pertinentes apprises : Langages de Programmation, Routage et
    Commutation, Conception Orientée Objet, Algorithmes, Base de Données
    Relationnelle, Structures de Données, Développement de Logiciels.
  \end{itemize*}
  \item
  \headerrow
    {\textbf{Institut Saint-Luc}}
    {\textbf{Tournai, Belgique}}
  \headerrow
    {\emph{École d'Art orientation Photographie}}
    {\emph{2015 -- 2018}}
  \begin{itemize*}
    \item
    CESS/BAC, Certificat de technicien photographe, Certificat de gestion de
    base.
    \item[]
    \rule{0pt}{2ex}\footnotesize photos at
    \href{http://www.victorhachard.fr}
    {victorhachard.fr}
  \end{itemize*}
\end{itemize}


% Projects section.
\hrule
\vspace{-0.4em}
\subsection*{Projets}
\begin{itemize}
  \parskip=0.1em

  \item
    \headerrow
        {\textbf{YouTube Playlist Checker}}
        {\emph{HTML/CSS/PHP/SQL}}
    \headerrow
      {\emph{Projet personnel}}
      {\emph{2018 -- 2020}}
    \begin{itemize*}
      \item
      Création de plusieurs applications web incluant \href{http://www.ypc.yt}
      {ypc.yt} : un site permettant d'archiver des vidéos YouTube.
      Apprentisage : PHP, SQL, structuration d'une de base de donnée,
      création d'une application web/API depuis zéro, utilisation de L'API Youtube.
      \item[]
      \rule{0pt}{2ex}\footnotesize source code at
      \href{http://www.github.com/VictorHachard/YPC\textunderscore2.0}
      {github.com/VictorHachard/YPC\textunderscore2.0}
    \end{itemize*}
    \item
    \headerrow
      {\textbf{Boby Is You}}
      {\emph{Java/JavaFX}}
    \headerrow
      {\emph{Projet scolaire}}
      {\emph{2017}}
      \begin{itemize*}
        \item
        Remake du jeu  Baba Is You, de Hempuli. Apprentisage : Java, JavaFX,
        gestion de fichiers, Apache Ant, concepts de programmation
        de base.
        \item[]
        \rule{0pt}{2ex}\footnotesize source code at
        \href{http://www.github.com/VictorHachard/BobyIsYou}
        {github.com/VictorHachard/BobyIsYou}
      \end{itemize*}
  \end{itemize}


% Skills section.
\hrule
\vspace{-0.4em}
\subsection*{Compétences}
\begin{indentsection}{\parindent}
\begin{description*}
  \item[Langues :]
  Français \emph{(langue maternelle)}, Anglais \emph{(capacité professionnelle
  complète)}.
 \item[Langages :] Java, PHP, Python, SQL, C, \Csharp, HTML,
  CSS,  JavaScript/jQuery, Shell, Bash, TeX.
  \item[Systèmes :] Windows, Windows Server, Linux.
  \item[Logiciels :] Atom, IntelliJ, Eclipse, Visual Studio Code, GitHub, Git,
  VirtualBox, Photoshop.
\end{description*}
\end{indentsection}


% Interests/About section.
\hrule
\begin{indentsection}{\parindent}
\begin{description*}
\item[Centres d'intérêt :]
  art, cuisine épicée, DIY, enseignement, hardware, jeux vidéo (factory building),
  microélectronique, musique (classique, rock classique, 80), natation, travails
  manuels et voyages. 
\item[À propos de moi :]
  j'ai envie d'apprendre tout ma vie, je suis ponctuel, fiable,
  bon en communication orale, autodidacte.
\end{description*}
\end{indentsection}


% Metalink.
\begin{center}
\definecolor{gray}{rgb}{0.5,0.5,0.5}
\footnotesize \latex source code at
\href{http://www.github.com/VictorHachard/resume}
{github.com/VictorHachard/resume} - fork from
\href{http://www.github.com/nixpulvis}
{github.com/nixpulvis}
\end{center}

\end{document}
